\documentclass[12pt]{article}
\usepackage[spanish]{babel}
\usepackage[utf8]{inputenc}
\usepackage{graphicx}
\usepackage{xcolor}
\usepackage{geometry}
\usepackage{amsmath}
\usepackage[section]{placeins}
\usepackage{hyperref}
\hypersetup{
    colorlinks = true,
    linkcolor = blue,
    urlcolor = cyan
}

\geometry{a4paper, margin=2.5cm}

\title{Documentación de API de Gestión de Restaurante}
\author{Daniel Calvar Cruz}
\date{\today}

\begin{document}

\maketitle

\tableofcontents
\newpage

\section{Introducción}
\hyperlink{anchor-indice}{\textbf{Volver}}\\

Esta API ofrece un servicio CRUD sobre una BBDD con dos colecciones, grupos y usuarios.\\

La API está construida con Express.js y utiliza MongoDB como base de datos.

\section{Estructura}

\subsection{Rutas Principales}
La API está organizada en dos apartados:

\begin{itemize}
    \item \textbf{Grupo}
    \item \textbf{Usuario}
\end{itemize}

\subsection{Modelos de Datos}

\subsubsection{Modelo Grupo}
\begin{verbatim}
({
  nombre: {
            type: String
          },
  descripcion: {
            type: String
          },
  activo: {
            type: Boolean
          }
});
\end{verbatim}

\subsubsection{Modelo Usuario}
\begin{verbatim}
({
  nombre: {
            type: String
          },
  apellido1: {
            type: String
          },
  apellido2: {
            type: String
          },
  edad: {
            type: Number
          }
});
\end{verbatim}

\clearpage

\section{Base de Datos}
\hyperlink{anchor-indice}{\textbf{Volver}}\\

MongoDB (Atlas).\\

{\textbf{Estructura}}
\begin{itemize}
  \item grupos: contiene los estados de los grupos.
  \item usuarios: contiene los estados de los usuarios.
\end{itemize}
\clearpage

\section{Endpoints}
\hyperlink{anchor-indice}{\textbf{Volver}}\\

\section{Endpoints}
\hyperlink{anchor-indice}{\textbf{Volver}}\

\subsection{Endpoints POST}

\subsubsection{POST /api/usuarios/create}
\textbf{Descripción:} Crea un nuevo usuario en el sistema.\

\textbf{Body:}
\begin{verbatim}
{
"nombre": "string",
"apellido1": "string",
"apellido2": "string",
"edad": "number"
}
\end{verbatim}

\textbf{Respuestas:}
\begin{itemize}
\item \textbf{200}: Usuario creado exitosamente
\item \textbf{400}: Error en los datos proporcionados
\item \textbf{500}: Error interno del servidor
\end{itemize}

\subsubsection{POST /api/usuarios/read}
\textbf{Descripción:} Busca usuarios según un filtro específico.\

\textbf{Body:}
\begin{verbatim}
{
"filtroKey": "string",
"filtroValue": "string|number"
}
\end{verbatim}

\textbf{Ejemplos de uso:}
\begin{itemize}
\item Buscar por nombre: \verb|{"filtroKey": "nombre", "filtroValue": "Ana"}|
\item Buscar por edad: \verb|{"filtroKey": "edad", "filtroValue": 28}|
\end{itemize}

\textbf{Respuestas:}
\begin{itemize}
\item \textbf{200}: Información encontrada con datos
\item \textbf{404}: Usuarios no encontrados
\item \textbf{500}: Error interno del servidor
\end{itemize}

\subsubsection{POST /api/grupos/create}
\textbf{Descripción:} Crea un nuevo grupo en el sistema.\

\textbf{Body:}
\begin{verbatim}
{
"nombre": "string",
"descripcion": "string",
"activo": "boolean"
}
\end{verbatim}

\textbf{Respuestas:}
\begin{itemize}
\item \textbf{200}: Grupo creado exitosamente
\item \textbf{400}: Error en los datos proporcionados
\item \textbf{500}: Error interno del servidor
\end{itemize}

\subsubsection{POST /api/grupos/read}
\textbf{Descripción:} Busca grupos según un filtro específico.\

\textbf{Body:}
\begin{verbatim}
{
"filtroKey": "string",
"filtroValue": "string|boolean"
}
\end{verbatim}

\textbf{Ejemplos de uso:}
\begin{itemize}
\item Buscar por nombre: \verb|{"filtroKey": "nombre", "filtroValue": "Administradores"}|
\item Buscar por estado: \verb|{"filtroKey": "activo", "filtroValue": true}|
\end{itemize}

\textbf{Respuestas:}
\begin{itemize}
\item \textbf{200}: Información encontrada con datos
\item \textbf{404}: Grupos no encontrados
\item \textbf{500}: Error interno del servidor
\end{itemize}

\clearpage

\subsection{Endpoints PUT}
\hyperlink{anchor-indice}{\textbf{Volver}}\

\subsubsection{PUT /api/usuarios/update}
\textbf{Descripción:} Actualiza la información de un usuario existente.\

\textbf{Body:}
\begin{verbatim}
{
"_id": "string",
"nombre": "string",
"apellido1": "string",
"apellido2": "string",
"edad": "number"
}
\end{verbatim}

\textbf{Respuestas:}
\begin{itemize}
\item \textbf{200}: Usuario actualizado exitosamente
\item \textbf{400}: Error en los datos proporcionados
\item \textbf{404}: Usuario no encontrado
\item \textbf{500}: Error interno del servidor
\end{itemize}

\subsubsection{PUT /api/grupos/update}
\textbf{Descripción:} Actualiza la información de un grupo existente.\

\textbf{Body:}
\begin{verbatim}
{
"_id": "string",
"nombre": "string",
"descripcion": "string",
"activo": "boolean"
}
\end{verbatim}

\textbf{Respuestas:}
\begin{itemize}
\item \textbf{200}: Grupo actualizado exitosamente
\item \textbf{400}: Error en los datos proporcionados
\item \textbf{404}: Grupo no encontrado
\item \textbf{500}: Error interno del servidor
\end{itemize}

\clearpage

\subsection{Endpoints DELETE}
\hyperlink{anchor-indice}{\textbf{Volver}}\

\subsubsection{DELETE /api/usuarios/delete}
\textbf{Descripción:} Elimina un usuario del sistema.\

\textbf{Body:}
\begin{verbatim}
{
"_id": "string"
}
\end{verbatim}

\textbf{Respuestas:}
\begin{itemize}
\item \textbf{200}: Usuario eliminado exitosamente
\item \textbf{400}: Error en los datos proporcionados
\item \textbf{404}: Usuario no encontrado
\item \textbf{500}: Error interno del servidor
\end{itemize}

\subsubsection{DELETE /api/grupos/delete}
\textbf{Descripción:} Elimina un grupo del sistema.\

\textbf{Body:}
\begin{verbatim}
{
"_id": "string"
}
\end{verbatim}

\textbf{Respuestas:}
\begin{itemize}
\item \textbf{200}: Grupo eliminado exitosamente
\item \textbf{400}: Error en los datos proporcionados
\item \textbf{404}: Grupo no encontrado
\item \textbf{500}: Error interno del servidor
\end{itemize}

\clearpage

\subsection{Endpoints GET}
\hyperlink{anchor-indice}{\textbf{Volver}}\

\subsubsection{GET /api/usuarios/readall}
\textbf{Descripción:} Obtiene todos los usuarios del sistema.\

\textbf{Parámetros:} Ninguno\

\textbf{Respuestas:}
\begin{itemize}
\item \textbf{200}: Lista de usuarios obtenida exitosamente
\item \textbf{404}: No hay usuarios registrados
\item \textbf{500}: Error interno del servidor
\end{itemize}

\subsubsection{GET /api/grupos/readall}
\textbf{Descripción:} Obtiene todos los grupos del sistema.\

\textbf{Parámetros:} Ninguno\

\textbf{Respuestas:}
\begin{itemize}
\item \textbf{200}: Lista de grupos obtenida exitosamente
\item \textbf{404}: No hay grupos registrados
\item \textbf{500}: Error interno del servidor
\end{itemize}


\clearpage

\section{Manejo de Errores}
\hyperlink{anchor-indice}{\textbf{Volver}}\\

La API utiliza un sistema consistente de respuestas:

\begin{verbatim}
{
  "type": "success|failure",
  "message": "string descriptivo",
  "data": "object (opcional)"
}
\end{verbatim}

\section{Ejemplos de Uso}

\begin{verbatim}
POST http://localhost:5000/api/usuarios/create
Content-Type: application/json

{
  "nombre": "Carlos",
  "apellido1": "Martínez",
  "apellido2": "Rodríguez",
  "edad": 35
}

POST http://localhost:5000/api/grupos/create
Content-Type: application/json

{
  "nombre": "Desarrolladores",
  "descripcion": "Equipo de desarrollo de software",
  "activo": true
}

POST http://localhost:5000/api/usuarios/read
Content-Type: application/json

{
  "filtroKey": "nombre",
  "filtroValue": "Carlos"
}

POST http://localhost:5000/api/grupos/read
Content-Type: application/json

{
  "filtroKey": "activo", 
  "filtroValue": true
}
\end{verbatim}


\clearpage

\section{Consideraciones Técnicas}
\hyperlink{anchor-indice}{\textbf{Volver}}\\

\section{Códigos de Estado HTTP}

\begin{itemize}
    \item \textbf{200 OK}: Operación exitosa
    \item \textbf{400 Bad Request}: Error en los datos enviados
    \item \textbf{404 Not Found}: Recurso no encontrado
    \item \textbf{500 Internal Server Error}: Error del servidor
\end{itemize}

\section{Stack tecnológico}

\begin{itemize}
    \item \textbf{Versión Node.js:} 22.14
    \item \textbf{Base de datos:} MongoDB Atlas
    \item \textbf{Dependencias NPM:}
    \begin{itemize}
        \item \textbf{cors:} 2.8.5
        \item \textbf{dotenv:} 16.4.7
        \item \textbf{express:} 4.21.1
        \item \textbf{mongodb:} 6.12.0
        \item \textbf{mongoose:} 8.13.2
        \item \textbf{nodemon:} 3.1.10  
    \end{itemize}
\end{itemize}

\section{Variables de entorno (.env)}

    \begin{itemize}
    \item \texttt{PORT}: Puerto en el que se ejecutará el servidor.
    \item \texttt{MONGO\_INITDB\_ROOT\_USERNAME}: Nombre del usuario de la BBDD.
    \item \texttt{MONGO\_INITDB\_ROOT\_PASSWORD}: Password de la BBDD.
    \item \texttt{MONGO\_INITDB\_DATABASE}: Nombre de la BBDD.
    \end{itemize}

\end{document}